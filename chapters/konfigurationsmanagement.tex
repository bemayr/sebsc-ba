\chapter{Konfigurationsmanagement}
Im Zusammenhang mit Docker und dem Softwareentwicklungsprozess werden oftmals zahlreiche weitere Werkzeuge genannt, die ähnliche Einsatzgebiete haben, oder in Kombination mit Docker den Prozess erheblich verbessern können. In den folgenden Abschnitten werden die bekanntesten und am weitest verbreiteten Werkzeuge und deren jeweilige Einsatzzwecke vorgestellt. Zusätzlich werden exemplarische Szenarien und die mögliche Kombination mit Docker aufgezeigt.

Die vorgestellten Werkzeuge dienen der Konfiguration der Produkt- und Entwicklungsumgebungen.
Diese Konfigurationen ließen sich auch manuell vornehmen, doch damit treten erhebliche Probleme im Sinne der Testbarkeit, Reproduzierbarkeit und Automatisierung auf.
Ein noch größeres Problem stellt das implizite Wissen der Entwickler dar, das bei einer händischen Konfiguration nur verbal weitergegeben wird.
Auch dokumentierte Handbücher sind problematisch hinsichtlich der fehlerlosen Reproduzierbarkeit, weshalb die Idee der Infrastrukturverwaltung in Versionskontrollsystemen entstanden ist.  

\section{Infrastructure as Code}

\section{Werkzeuge}
\subsection{Vagrant}
\subsection{Chef}
\subsection{Puppet}
\subsection{Ansible}
\subsection{Salt}
\subsection{Packer}
\subsection{Habitat}

\section{Kombinationsmöglichkeiten}
