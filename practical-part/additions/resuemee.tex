\chapter{Resümee}
\label{cha:resume}
Um einen kurzen Überblick über die Arbeit zu geben, wird diese in diesem Kapitel zusammengefasst.
Weiters werden die daraus gewonnenen Erkenntnisse dargestellt, sowie ein Blick in die Zukunft gewagt.

\section{Zusammenfassung}
\label{sec:summary}
Die Anzahl und Komplexität der Werkzeuge in der Frontend-Webentwicklung im Jahr 2017 stellt ein großes Problem für Firmen da, die an vielen Projekten gleichzeitig arbeiten.
Dabei sind nicht die neuen Werkzeuge problematisch, vielmehr ist es aufgrund von Versionsproblemen beinahe unmöglich, unterschiedlich alte Projekte auf einem Entwicklerrechner zu warten.

Durch die Verwendung von Docker kann ein Großteil der benötigten Werkzeuge mit dem Projekt mitversioniert werden, wodurch die Wartbarkeit von Anwendungen stark erhöht wird.
Der softaware/webdev-Container bietet die Basis dazu und liefert mit der Dokumentation einen Überblick über die Möglichkeiten und Probleme beim Einsatz von Containern als Entwicklerwerkzeuge.

\section{Erkenntnisse}
\label{sec:findings}
Die in der Arbeit beschriebene Verbesserung der Webentwicklung müsste gar nicht existieren, wenn die diversen Werkzeuge rückwärtskompatibel und korrekt versioniert wären.
Da sich das allerdings im Nachhinein nicht mehr ändern lässt, löst ein Container in diesem Fall viele Probleme.

Durch die Erstellung von Startskripten kann Docker soweit abstrahiert werden, dass es auch von Entwicklern, die noch nichts mit Containertechnologien zu tun hatten, problemlos eingesetzt werden kann.
Zusätzlich führt die Verwendung des Containers zu einem einheitlichen Werkzeugset und einer daraus resultierenden geringeren Hemmschwelle beim Projekteinstieg.

\section{Ausblick}
\label{sec:outlook}
Die Verwendung von Containern als Entwicklerwerkzeuge zeigt, genau in einem Beispiel wie diesem, ihre Stärken.
In einer idealen Welt gäbe es allerdings keine Versionskonflikte zwischen Node.js-Paketen und die Notwendigkeit für den softaware/webdev-Container wäre nicht gegeben.
