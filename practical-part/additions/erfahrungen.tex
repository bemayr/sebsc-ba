\chapter{Erfahrungen}
\label{cha:experience}
Die Einführung einer neuen Technologie in ein Unternehmen ist immer eine aufregende Angelegenheit.
Einerseits sind neue Technologien interessant und spannend, doch bei genauerer Betrachtung wird sofort skeptisch hinterfragt, ob das oftmals mühsame Einarbeiten in die neue Technologie auch wirklich den erwarteten Nutzen bringt.
Besonders bei einer neuen Art der Technologie, wie Containertechnologien es sind, kann die anfängliche Euphorie sehr schnell der Frustration weichen.
Wenn allerdings ein Problem existiert, dessen Lösung mithilfe dieser neuen Technologie realistisch erscheint, ist wesentlich weniger Überzeugungskraft notwendig.

Nach den gewonnenen Erfahrungen aus meiner theoretischen Bachelorarbeit, war es für mich sehr interessant, im Praktikum wiederkehrende Probleme meiner Kollegen zu analysieren und versuchen einen völlig anderen Lösungsweg zu suchen.
Solche Aufgaben gefallen mir besonders gut, da man dadurch meist lästige, unlösbar geglaubte Probleme löst, die allen Beteiligten viel Arbeit und Zeit sparen.

Nach gut drei Wochen im Praktikum war die Idee des softaware/webdev-Containers geboren, nachdem sich im Büro die Probleme mit npm und Node.js häuften.

Erst die genauere Recherche über die Werkzeuge der Webentwicklung hat mir das volle Ausmaß der Werkzeugvielfalt in der Webentwicklung vor Augen geführt.
Aufgrund des Node.js-Basisimages war die erste Version des Containers sehr schnell fertig und konnte getestet werden.
Der Einsatz in realen Projekten und das ständige Feedback der Kollegen führt zu einer stetigen Weiterentwicklung des Containers.

Dass der Quelltext des Containers veröffentlicht wurde, führte auch schon bei diversen Events zu interessanten Diskussionen und hat mir die Vor- und Nachteile der Open-Source-Entwicklung aufgezeigt.
Auch das Veröffentlichen eines Blogartikels im Blog der Firma softaware gmbh war eine sehr lehrreiche Erfahrung.
Besonders als der Container beim Start eines neuen Webprojektes beinahe reibungslos in dieses integriert wurde, war es ein sehr gutes Gefühl, das Ergebnis meiner praktischen Bachelorarbeit in realen Projekten im Einsatz zu sehen.
