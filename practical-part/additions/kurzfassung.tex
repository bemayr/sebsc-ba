\addchap{Kurzfassung}
Aufgrund der immensen Geschwindigkeit in der sich die Webentwicklung vorwärtsbewegt, erlangen Anwendungen bereits in wenigen Monaten den Legacy-Status.
Eines der größten Probleme stellt die Instandhaltung der benötigten Werkzeuge dar.
Bei der Entwicklung unterschiedlich alter Projekte kommt es unweigerlich zu Problemen durch die unterschiedlichen Versionen der Werkzeuge.

Im Rahmen einer Analyse des Status der Frontend-Webentwicklung werden die aktuell relevanten Werkzeuge gezeigt und eine Begründung für diesen Trend gesucht.
Bevor auf den softaware/webdev-Container eingegangen wird, werden die Vor- und Nachteile von containerbasierten Entwicklertools dargestellt.
Zusätzlich zu den Implementierungsdetails des softaware/webdev-Containers werden dessen Einsatzbereiche, die Motivation für die Entwicklung sowie die Besonderheiten und Probleme gezeigt.
