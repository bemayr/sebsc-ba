\addchap{Kurzfassung}
Softwareentwicklung in Zeiten von Node.js, ASP.NET Core und Microservices impliziert die Verwendung einer Menge an Werkzeugen.
Die Hauptaufgabe des Softwareentwicklers ist es allerdings, Software zu entwickeln und nicht den Großteil seiner Zeit mit dem Erlernen und der Konfiguration dieser Werkzeuge zu verbringen.

Mit Docker wurde eine Plattform zur Vereinfachung der Softwareentwicklung auf Basis von Containervirtualisierung geschaffen, deren Ziele eine möglichst einfache Verwendung und Plattformunabhängigkeit sind.
Anhand der Geschichte der Containervirtualisierung wird in dieser Arbeit ein Vergleich zu herkömmlichen Virtualisierungsmöglichkeiten gezogen sowie die Funktionsweise und notwendigen Systemvoraussetzungen für die Verwendung von Containern dargestellt.
Nach einer Übersicht über bestehende Werkzeuge des Konfigurationsmanagements wird für Softwareentwickler eine Einführung in die Verwendung von Docker zu Entwicklungszwecken gegeben, wobei besonders auf Best-Practices und Integrationsmöglichkeiten in den Entwicklungsprozess Wert gelegt wird.

Am Beispiel verschiedener Anwendungsfälle wird gezeigt, wie der Entwicklungsprozess mithilfe von Docker stabiler und einheitlicher gestaltet werden kann, um das erklärte Ziel, einen produktiveren und angenehmeren Workflow für Softwareentwickler, zu erreichen.

Aktuelle Entwicklungen wie Eclipse Che und Windows Container zeigen, dass Containervirtualisierung im Allgemeinen und Docker im Speziellen eine große Zukunft bevorsteht.
