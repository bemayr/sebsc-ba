%%% --- TODO ---
%% Aufgabenstellung
%%%
\chapter{Einleitung}
\section{Motivation}
Die tägliche Arbeit als Softwareentwickler beinhaltet die Verwendung von zahlreichen Werkzeugen.
Besonders bei der Entwicklung unterschiedlicher Plattformen wird die Kombination der zahlreichen Build-Tools, Compilern, Transpilern, Task-Runnern, Paketmanagern und Datenbanken schnell eine fragile Kombination an Kommandos.
Projekte entwickeln sich weiter und neue Technologien erscheinen, schnell wird aus einem \texttt{npm run dev} ein \texttt{npm start} oder \texttt{mvn build} bricht aufgrund einer falschen Java-Version ab.
Die große Vielfalt der Werkzeuge, besonders im Web-Bereich, verlangt vom Entwickler, dass er sich für jedes Projekt das Wissen für ein neues Werkzeugset aneignet und dieses auch zu verwenden weiß.

Seit 2014 verspricht Docker mit dem Werbespruch \emph{Build, Ship, Run} eine Verbesserung des Softwareentwicklungsprozesses mithilfe von Containervirtualisierung.
Welche Möglichkeiten der Automatisierung für den Softwareentwickler möglich sind, inwieweit ein plattformunabhängiger Entwicklungsprozess möglich ist und ob sich durch den Einsatz von Docker die Lernkurve eines Entwicklers beim Einstieg in ein Softwareprojekt verringern lässt, sind motivierende Gründe für diese Bachelorarbeit.

\section{Ziel}
\textbf{Notizen}
Durch exemplarisches Lösen einiger Probleme mit denen Softwareentwickler häufig konfrontiert sind, soll dem Leser die Verwendung und Flexibilität von Docker gezeigt werden.
Weiters sollen Softwareentwickler automatisierbare Probleme erkennen und verstehen ob und wie Docker in diesen Fällen zu einer zeit- und ressourcensparenden Lösung beitragen könnte.

Zur Erreichung dieser entwicklergesteuerten Sicht auf Docker werden im theoretischen Teil der Arbeit verschiedene Virtualisierungstechniken verglichen und in der Funktionsweise und dem Einsatzzweck von Docker abgegrenzt.
Die grundlegenden, für den Entwickler notwendigen Konzepte von Docker werden ebenso geschildert wie eine Übersicht und Erklärung der gemeinsam mit Docker am häufigsten erwähnten Werkzeuge.

% Begriffe und Technologien im Rahmen von Docker kurz zu erklären, damit es einen Überblick über die Tools gibt und sich nicht jeder Einlesen muss, wofür welches Tool verwendet werden kann und warum, oder warum es nicht zum momentanen Anwendungsfall passt.

% \section{Glossar}
% \begin{description}
%    \item[Vollvirtualisierung]
%    \item [DevOps]
% \end{description}
