\addchap{Abstract}
Using technologies like Node.js, ASP.NET Core and Microservices results in an immense variety of tools for developers.
But investing hours of work studying, tweaking and configuring the tools should not be the primary task of a software developer.
It must be the aim to use less time adapting the tools and more time actually developing the product itself.

With the release of Docker in 2014, using containers in software development got a lot easier.
Docker's main goals are platform independent usage and the highest possible ease of use.
The first part of the thesis targets the difference between container virtualization and conventional virtualization technologies as well as tools for configuration management that are comparable to Docker.

The internals of Docker, further products of the Docker ecosystem and their use-cases are described in the second part of this thesis.
A short introduction into the usage of Docker for developers is given, highlighting possibilities to integrate Docker into the software development process and focusing on industry-proven best practices.

Various use cases illustrate how a better development experience can be achieved by evolving a consistent and more stable development process.
