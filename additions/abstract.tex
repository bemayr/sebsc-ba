\addchap{Abstract}
When using technologies like Node.js, ASP.NET Core and Microservices developers have to cope with an immense variety of tools.
But investing hours of work studying, tweaking and configuring the tools should not be the primary task of a software developer.
It must be the aim to use less time adapting the tools and more time actually developing the product itself.

With the release of Docker in 2014, using containers in software development became a lot easier.
Docker's main goals are platform independence and the highest possible ease of use.
The thesis targets the difference between container virtualization and conventional virtualization technologies based on their histories as well as explaining the system requirements for Docker and how it works.
Commonly used configuration management tools are described, before a short introduction into the usage of Docker for developers is given, highlighting possibilities to integrate Docker into the software development process and focusing on industry-proven best practices.

Various use cases illustrate how a better development experience can be achieved by evolving a consistent and more stable development process.

Latest developments like Eclipse Che and Windows Containers show that container virtualization and Docker in particular are going to affect the way we write software.
