\addchap{Abstract}
Using technologies like Node.js, ASP.NET Core and Microservices results in an immense toolset for developers.
But investing hours of work studying, tweaking and configuring the toolset is not the primary task of a software developer.
It would be better if he uses less time adapting his tools and more time actually developing the product itself.

With the release of Docker in 2014 using Containers in software development got a easier. Docker's main goals are platform independent usage and the highest possible ease of use.
The first part of the thesis targets the difference between container virtualization and conventional virtualization technologies as well as other tools like Docker for configuration management.

The internals of Docker, further products of the Docker ecosystem and their use-cases are described in the second part.
With this in mind, a short introduction into the usage of Docker for developers is given, highlighting possibilites to integrate Docker into the software development process and focusing on industry-proven best practices.

Various use cases showcase how a better development experience can be achieved by evolving a consistent and more stable development process. 
