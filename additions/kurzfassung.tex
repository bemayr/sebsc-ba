\addchap{Kurzfassung}
Softwareentwicklung in Zeiten von Node.js, ASP.NET Core und Microservices impliziert die Verwendung einer Menge an Werkzeugen.
Die Aufgabe des Softwareentwicklers ist allerdings Software zu entwickeln und nicht den Großteil seiner Zeit mit dem Erlernen und der Konfiguration dieser Werkzeuge zu verbringen.

Mit Docker wurde eine Plattform zur Vereinfachung der Softwareentwicklung auf Basis von Containervirtualisierung geschaffen, deren Ziele eine möglichst einfache Verwendung und Plattformunabhängigkeit sind.
Welche Unterschiede zu herkömmlichen Virtualisierungskonzepten existieren und welche vergleichbaren Werkzeuge es zum Konfigurationsmanagement gibt, wird im ersten Teil der Arbeit untersucht.

Der zweite Teil analysiert den Aufbau von Docker.
Weiters wird eine Übersicht über die Produkte des Docker-Ökosystems und deren Einsatzzweck geboten.
Für Softwareentwickler wird eine Einführung in die Verwendung von Docker zu Entwicklungszwecken gegeben, wobei besonders auf Best-Practices und Möglichkeiten zur Integration von Docker in den Entwicklungsprozess eingegangen wird.

Anhand verschiedener Anwendungsfälle wird gezeigt, wie der Entwicklungsprozess mithilfe von Docker stabiler und einheitlicher gestaltet werden kann, um das erklärte Ziel, eine Steigerung der Developer Experience, zu erreichen.
